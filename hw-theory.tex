\documentclass[10pt,a4paper,oneside]{article}
\usepackage[utf8]{inputenc}
\usepackage[english,russian]{babel}
\usepackage{amsmath}
\usepackage{amsthm}
\usepackage{amssymb}
\usepackage{enumerate}
\usepackage{stmaryrd}
\usepackage{cmll}
\usepackage{mathrsfs}
\usepackage{hyperref}
\usepackage[left=2cm,right=2cm,top=2cm,bottom=2cm,bindingoffset=0cm]{geometry}
\usepackage{proof}
\usepackage{tikz}
\usepackage{multicol}

\makeatletter
\newcommand{\dotminus}{\mathbin{\text{\@dotminus}}}

\newcommand{\@dotminus}{%
  \ooalign{\hidewidth\raise1ex\hbox{.}\hidewidth\cr$\m@th-$\cr}%
}
\makeatother

\usetikzlibrary{arrows,backgrounds,patterns,matrix,shapes,fit,calc,shadows,plotmarks}

\newtheorem{definition}{Определение}
\begin{document}

\begin{center}{\Large\textsc{\textbf{Теоретические домашние задания}}}\\
             \it Теория типов, ИТМО, М3235-М3239, осень 2022 года\end{center}

\section*{Домашнее задание №1: <<вводная лекция>>}

\begin{enumerate}

\item Напомним определения с лекций:

\begin{tabular}{lll}
Обозначение & лямбда-терм & название\\\hline
$T$ & $\lambda a.\lambda b.a$ & истина\\
$F$ & $\lambda a.\lambda b.b$ & ложь\\
$Not$ & $\lambda x.x\ F\ T$ & отрицание\\
$And$ & $\lambda x.\lambda y.x\ y\ F$ & конъюнкция
\end{tabular}

Постройте лямбда-выражения для следующих булевских выражений:
\begin{enumerate}
\item Штрих Шеффера (<<и-не>>)
\item Стрелка Пирса (<<или-не>>)
\item Мажоритарный элемент от трёх аргументов (результат <<истина>>, если истинны не менее двух аргументов)
\end{enumerate}

\item Напомним определения с лекций:

$$f^{(n)}\ X ::= \left\{\begin{array}{ll} X, & n=0\\
                                f^{(n-1)}\ (f\ X), & n>0\end{array}\right.$$

\begin{center}\begin{tabular}{lll}
Обозначение & лямбда-терм & название\\\hline
$\overline{n}$ & $\lambda f.\lambda x.f^{(n)}\ x$ & чёрчевский нумерал\\
$(+1)$ & $\lambda n.\lambda f.\lambda x.n\ f\ (f\ x)$ & прибавление 1\\
$IsZero$ & $\lambda n.n\ (\lambda x.F)\ T$ & проверка на 0
\end{tabular}\end{center}

\begin{center}\begin{tabular}{lll}
Обозначение & лямбда-терм & название\\\hline
$MkPair$ & $\lambda a.\lambda b.(\lambda x.x\ a\ b)$ & создание пары\\
$PrL$ & $\lambda p.p\ T$ & левая проекция\\
$PrR$ & $\lambda p.p\ F$ & правая проекция\\\hline
$Case$ & $\lambda l.\lambda r.\lambda c.c\ l\ r$ & case для алгебраического типа\\
$InL$ & $\lambda l.(\lambda x.\lambda y.x\ l)$ & левая инъекция\\
$InR$ & $\lambda r.(\lambda x.\lambda y.y\ r)$ & правая инъекция\\
\end{tabular}\end{center}

Используя данные определения, постройте выражения для следующих операций над числами:

\begin{enumerate}
\item Вычитание 1
\item Вычитание
\item Деление
\item Сравнение двух чисел ($IsLess$) — истина, если первый аргумент меньше второго
(могут потребоваться пары и/или вычитания)
\item Делимость
\end{enumerate}

\item Проредуцируйте выражение и найдите его нормальную форму: 
\begin{enumerate}
\item $\overline{2}\ \overline{2}$
\item $\overline{2}\ \overline{2}\ \overline{2}$
\item $\overline{2}\ \overline{2}\ \overline{2}\ \overline{2}\ \overline{2}\ \overline{2}\ \overline{2}$
\end{enumerate}

\item Напомним определение Y-комбинатора: $\lambda f.(\lambda x.f\ (x\ x))\ (\lambda x.f\ (x\ x))$.
\begin{enumerate}
\item Покажите, что выражение $Y\ f$ не имеет нормальной формы;
\item Покажите, что выражение $Y\ (\lambda f.\overline{0})$ имеет нормальную форму.
\item Покажите, что выражение $Y\ (\lambda f.\lambda x.(IsZero\ x)\ \overline{0}\ (f\ Minus1\ x))\ 2$ имеет нормальную форму.
\item Какова нормальная форма выражения $Y\ (\lambda f.\lambda x.(IsZero\ x)\ \overline{0}\ ((+1)\ (f\ Minus1\ x)))\ \overline{n}$?
\item Какова нормальная форма выражения $Y\ (\lambda f.\lambda x.(IsZero\ x)\ \overline{1}\ (Mul2\ (f\ Minus1\ x)))\ \overline{n}$?
\item Определите с помощью $Y$-комбинатора функцию для вычисления $n$-го числа Фибоначчи.
\end{enumerate}

\item Определим на языке Хаскель следующую функцию: \verb!show_church n = show (n (+1) 0)!
Убедитесь, что \verb!show_church (\f -> \x -> f (f x))! вернёт 2. 
Пользуясь данным определением и его идеей, реализуйте следующие функции:

\begin{enumerate}
\item \verb!int_to_church! --- возвращает чёрчевский нумерал (т.е. функцию от двух аргументов) по целому числу.
Каков точный тип результата этой функции?
\item сложение двух чёрчевских нумералов.
\item умножение двух чёрчевских нумералов.
\item можно ли определить вычитание 1 и вычитание? Что получается, а что --- нет?
\end{enumerate}

\item На лекции было использовано понятие свободы для подстановки. 
\begin{enumerate}
\item Найдите лямбда-выражение, которое при однократной редукции требует переименования связанных переменных
(редукция невозможна без переименования). 
\item Заметим, что даже если мы запретим использовать одни и те же переменные в разных лямбда-абстракциях,
это не будет решением проблемы переименований. Предложите лямбда-выражение, в котором (а) все лямбда-абстракции
указаны по разным переменным; но (б) через некоторое количество редукций потребуется переименование
связанных переменных.
\end{enumerate}

\item Дадим определение: комбинатор --- лямбда-выражение без свободных переменных.

Также напомним определение:
$$\begin{array}{l}
S := \lambda x.\lambda y.\lambda z.x\ z\ (y\ z)\\
K := \lambda x.\lambda y.x\\
I := \lambda x.x
\end{array}$$

Известна теорема о том, что для любого комбинатора $X$ можно найти выражение $P$
(состоящее только из скобок, пробелов и комбинаторов $S$ и $K$), что $X =_\beta P$.
Будем говорить, что комбинатор $P$ \emph{выражает} комбинатор $X$ в базисе $SK$.

Выразите в базисе $SK$:
\begin{enumerate}
\item $F = \lambda x.\lambda y.y$
\item $\overline{1}$
\item $Not$
\item $Xor$
\item $InL$
\end{enumerate}

\item Чёрчевские нумералы соответствуют натуральным числам в аксиоматике Пеано.
\begin{enumerate}
\item Предложите <<двоичные нумералы>> --- способ кодирования чисел, аналогичный двоичной системе 
(такой, при котором длина записи числа соответствует логарифму числового значения).
\item Предложите реализацию функции (+1) в данном представлении.
\item Предложите реализацию лямбда-выражения преобразования числа из двоичного нумерала в чёрчевский.
\item Предложите реализацию функции сложения в данном представлении.
\item Предложите реализацию функции вычитания в данном представлении.
\item Какова вычислительная сложность арифметопераций с двоичными нумералами?
\end{enumerate}

\end{enumerate}

\section*{Домашнее задание №2: ещё о бестиповом лямбда-исчислении}

\begin{enumerate}
\item Бесконечное количество комбинаторов неподвижной точки. Дадим следующие определения
$$\begin{array}{l}
L := \lambda abcdefghijklmnopqstuvwxyzr.r(thisisafixedpointcombinator)\\
R := LLLLLLLLLLLLLLLLLLLLLLLLLL\end{array}$$
В данном определении терм $R$ является комбинатором неподвижной точки: каков бы ни был терм
$F$, выполнено $R\ F =_\beta F\ (R\ F)$.
\begin{enumerate}
\item Докажите, что данный комбинатор --- действительно комбинатор неподвижной точки.
\item Пусть в качестве имён переменных разрешены русские буквы. Постройте аналогичное выражение
по-русски: с 33 параметрами и осмысленной русской фразой в терме $L$; покажите, что оно является
комбинатором неподвижной точки.
\end{enumerate}

\item Напомним определение аппликативного порядка редукции:
редуцируется самый левый из самых вложенных редексов. Например, в случае выражения
$(\lambda x.I\ I)\ (\lambda y.I\ I)$ самые вложенные редексы --- применения $I\ I$:

$$(\lambda x.\underline{I\ I})\ (\lambda y.\underline{I\ I})$$

и надо выбрать самый левый из них:

$$(\lambda x.\underline{I\ I})\ (\lambda y.I\ I)$$
\begin{enumerate}
\item Проведите аппликативную редукцию выражения $2\ 2$.
\item Постройте выражение, использующее $Y$-комбинатор для вычисления факториала. 
Возможно ли его аппликативное вычисление, или оно не сможет завершиться?
\item Найдите лямбда-выражение, которое редуцируется медленнее при нормальном порядке редукции,
чем при аппликативном, даже при наличии мемоизации.
\end{enumerate}

\item Будем говорить, что выражение $A$ находится в \emph{слабой заголовочной нормальной форме} (WHNF),
если оно не имеет вид $A \equiv (\lambda x.P)\ Q$ (то есть, самый верхний терм его не является редексом).
Выражение находится в \emph{заголовочной нормальной форме} (HNF), когда его верхний терм --- не редекс и не лямбда-абстракция
с бета-редексами в теле.

Верно ли, что <<нормальность>> формы выражения может в процессе редукции только усиливаться
(никакая --- слабая заголовочная Н.Ф. --- заголовочная Н.Ф. --- нормальная форма)?

\item Заметим, что список в лямбда-выражении можно закодировать с помощью алгебраических типов.
Напишите лямбда-выражение для:
\begin{enumerate}
\item вычисления длины списка;
\item функции $map$ (построение нового списка из результатов применения функции к элементам старого);
\item суммы списка целых чисел.
\end{enumerate} 

\item Базис $SKI$ не единственный. Рассмотрим базис $BCKW$:
$$\begin{array}{l}B = \lambda x.\lambda y.\lambda z.x\ (y\ z)\\
C = \lambda x.\lambda y.\lambda z.x\ z\ y\\
K = \lambda x.\lambda y.x\\
W = \lambda x.\lambda y.x\ y\ y\end{array}$$

\begin{enumerate}
\item Покажите, что базис $BCKW$ позволяет выразить любое выражение из базиса $SKI$.
\item Покажите, что любое выражение из базиса $BCKW$ может быть выражено в базисе $SKI$.
\end{enumerate}
\end{enumerate}


\section*{Домашнее задание №3: просто-типизированное лямбда-исчисление}
\begin{enumerate}

\item Пусть фиксирован тип чёрчевского нумерала $\eta = (\alpha\rightarrow\alpha) \rightarrow (\alpha\rightarrow\alpha)$. 
Найдите тип для следующих конструкций и постройте доказательство:
\begin{enumerate}
\item $\overline{2}$ (покажите, что его тип --- $\eta$).
\item $(+1)$.
\item $Plus$.
\item $Mul$ (не каждая реализация умножения имеет тип в просто-типизированном лямбда исчислении;
вам требуется найти нужную)
\end{enumerate}

\item Имеет ли тип --- и какой:
\begin{enumerate}
\item операция вычитания 1 (выраженная через <<трюк зуба мудрости>>)? Общий ответ не требуется, 
достаточно рассмотреть какую-то одну реализацию.
\item операция вычитания ($\lambda m.\lambda n.m\ (-1)\ n$)?
\item операция возведения в степень ($Power ::= \lambda m.\lambda n.n\ m$)? 
\item функция $\lambda x.Power\ x\ x$?
\end{enumerate}

\item Каков тип:
\begin{enumerate}
\item комбинаторов $S$ и $K$;
\item истины и лжи.
\end{enumerate}

\item Рассмотрим полную интуиционистскую логику, с конъюнкцией, дизъюнкцией и ложью. Какой тип у следующих конструкций,
и какие правила вывода интуиционистской логики им соответстсвуют (ответ требует демонстрации корректности этих правил
для данных конструкций --- то есть вывод про тип результата применения правила должен всегда иметь место для выражений
соответствующего вида):
\begin{enumerate}
\item Упорядоченная пара (MkPair, PrL, PrR).
\item Алгебраический тип (InL: $\lambda x.\lambda a.\lambda b.a\ x$, InR: $\lambda x.\lambda a.\lambda b.b\ x$, Case).
\end{enumerate}

%\item Предложим альтернативные аксиомы для конъюнкции:
%
%$$\infer[\text{Введ. }\with]{\Gamma\vdash \alpha\with \beta}{\Gamma\vdash \alpha\ \ \ \Gamma\vdash \beta}\quad\quad
%  \infer[\text{Удал. }\with]{\Gamma\vdash \gamma}{\Gamma\vdash \alpha\with \beta\ \ \ \Gamma, \alpha, \beta\vdash \gamma}$$
%
%\begin{enumerate}                                                             
%\item Предложите лямбда-выражения, соответствующие данным аксиомам; поясните, как данные выражения 
%абстрагируют понятие <<упорядоченной пары>>.
%\item Выразите изложенные в лекции аксиомы конъюнкции через приведённые в условии.
%\item Выразите приведённые в условии аксиомы конъюнкции через изложенные в лекции.
%\end{enumerate}

\item Докажите лемму о редукции (subject reduction lemma): если $A \twoheadrightarrow_\beta B$ и $\vdash A: \tau$, то
$\vdash B: \tau$.

Верно ли обратное: если $A \twoheadrightarrow_\beta B$ и $\vdash B: \tau$, то
$\vdash A: \tau$?

\item Как мы уже разбирали, $\not\vdash x\ x:\tau$ в силу дополнительных ограничений
правила
$$\infer[x \notin FV(\Gamma)]{\Gamma, x:\tau\vdash x:\tau}{}$$

Найдите лямбда-выражение $N$, что $\not\vdash N:\tau$ в силу ограничений правила
$$\infer[x \notin FV(\Gamma)]{\Gamma \vdash \lambda x.N:\sigma\to\tau}{\Gamma, x:\sigma \vdash N:\tau}$$

\item Верно ли, что $S = B(BW)(BBC)$? Если нет, то как правильно?
\end{enumerate}

\section*{Домашнее задание №4: <<Изоморфизм Карри-Ховарда>>}

\begin{enumerate}
\item Предложим альтернативные аксиомы для конъюнкции:

$$\infer[\text{Введ. }\with]{\Gamma\vdash \alpha\with \beta}{\Gamma\vdash \alpha\ \ \ \Gamma\vdash \beta}\quad\quad
  \infer[\text{Удал. }\with]{\Gamma\vdash \gamma}{\Gamma\vdash \alpha\with \beta\ \ \ \Gamma, \alpha, \beta\vdash \gamma}$$

\begin{enumerate}
\item Предложите лямбда-выражения, соответствующие данным аксиомам; поясните, как данные выражения 
абстрагируют понятие <<упорядоченной пары>>.
\item Выразите изложенные в лекции аксиомы конъюнкции через приведённые в условии.
\item Выразите приведённые в условии аксиомы конъюнкции через изложенные в лекции.
\end{enumerate}

\item \emph{Вполне упорядоченным} множеством назовём такое линейно-упорядоченное отношением $(\prec)$ 
множество $S$ (и такой порядок назовём \emph{полным}), 
что какое бы ни было множество $U \subseteq S$, в $U$ найдётся наименьший элемент.
\begin{enumerate}
\item Покажите, что неотрицательные вещественные числа $[0,+\inf)$ --- не вполне упорядоченное множество.
Существуют ли конечные и счётные не вполне упорядоченные множества?
\item Определим лексикографический порядок на $\mathbb{N}^n$: положим, что 
$\langle a_1, a_2, \dots a_n \rangle \prec \langle b_1, b_2, \dots b_n$, если найдётся такой $k$,
что $a_1 = b_1$, ..., $a_{k-1} = b_{k-1}$, но $a_k < b_k$.
Покажите, что такой порядок --- полный.
\item Пусть $S$ вполне упорядочено отношением $(\prec)$, 
определим $a\succ b := b \prec a$. Пусть $a_1 \succ a_2 \succ a_3 \succ \dots$ --- 
строго монотонно убывающая последовательность значений из $S$. Покажите, что данная 
последовательность всегда имеет конечную длину.
\end{enumerate}

\item Поясним название <<алгебраические типы>> --- это семейство составных типов, 
позволяющих строить <<алгебраические>> выражения на типах:

\begin{tabular}{lll}
название & обозначение & алгебраический смысл\\\hline
тип-сумма, <<алгебраический>> & $\alpha\vee\beta$ & $\alpha+\beta$\\
тип-произведение, пара & $\alpha\with\beta$ & $\alpha\times\beta$\\
тип-степень, функция & $\alpha\to\beta$&$\beta^\alpha$
\end{tabular}

Название <<алгебраический>> закрепилось в первую очередь за типом-суммой (видимо потому,
что остальные типы имеют устоявшиеся названия), однако, может быть отнесено и к другим
типам.

Поясните <<типовый>> (программистский) смысл следующих алгебраических тождеств --- и постройте
программы на Хаскеле, их доказывающие:
\begin{enumerate}
\item $\gamma\times(\alpha+\beta) = \gamma\times\alpha + \gamma\times\beta$.
\item $\gamma^{\alpha\times\beta} = {(\gamma^\alpha)}^\beta$. Как называется данное тождество?
\item $\gamma^{\alpha+\beta} = \gamma^\alpha\times\gamma^\beta$.
\end{enumerate}

\item Напомним, что $\neg\alpha \equiv \alpha\rightarrow\bot$. Найдите лямбда-выражения, доказывающие:
\begin{enumerate}
\item Формулу де-Моргана $\neg(\alpha\vee\beta)\to\neg\alpha\with\neg\beta$.
\item Контрапозицию $(\alpha\to\beta)\to(\neg\beta\to\neg\alpha)$.
\item Закон исключённого третьего после применения теоремы Гливенко $\neg\neg(\alpha\vee\neg\alpha)$.
\end{enumerate}

\item Какие аксиомы соответствуют базису $BCKW$? Покажите, что аксиома, соответствующая $S$,
доказывается в этой аксиоматике.

\item Выразите в Хаскеле $Y$-комбинатор. Каков его тип?

\item Покажите, что типовая система Хаскеля противоречива.

\end{enumerate}

\section*{Домашнее задание №5: <<Реконструкция типа лямбда-выражений в просто-типизированном исчислении>>}
\begin{enumerate}
\item На лекции вводилась метрика для доказательства завершаемости алгоритма унификации: упорядоченная тройка $\langle x,y,z \rangle$,
где $x$ --- количество уравнений в разрешённой форме (уравнений вида $a = \theta$, причём $a$ входит в систему ровно один раз), 
$z$ --- количество уравнений типа $a=a$ и $\theta=b$. Смысл же параметра $y$ не был раскрыт.

Каким взять параметр для $y$, чтобы получившаяся метрика строго монотонно убывала при каждом применении правил унификации?

\item Применив алгоритм, рассказанный на лекции, найдите тип для комбинатора $S$.
\item Применив алгоритм, рассказанный на лекции, покажите отсутствие типа у $Y$-комбинатора.

\item Исчислением предикатов второго порядка назовём исчисление со следующим языком:
$$\Phi ::= x | (\Phi \rightarrow \Phi) | (\forall x.\Phi)$$
Содержательное отличие от исчисления высказываний --- наличие квантора всеобщности и правил вывода для его введения и удаления: 

    \[ \dfrac{\Gamma\vdash\phi}{\Gamma\vdash\forall p.\phi} (p\notin FV(\Gamma)) \qquad
        \dfrac{\Gamma\vdash\forall p.\phi}{\Gamma\vdash\phi[p:=\Theta]} \]

Докажите, что следующие связки могут быть выражены в таком исчислении (то есть, покажите, что сооветствующие формулы
удовлетворяют соответствующим правилам вывода для интуиционистского исчисления высказываний):
\begin{enumerate}
\item $\psi \with \varphi := \forall g.(\psi \rightarrow \varphi \rightarrow g) \rightarrow g$
\item $\psi \vee \varphi := \forall g.(\psi \rightarrow g) \rightarrow (\varphi \rightarrow g) \rightarrow g$
\item $\bot := \forall a.a$
\item $\exists a.\psi := \forall g.(\forall a.(\psi \rightarrow g)) \rightarrow g$

Для квантора существования правила вывода следующие:
    \[ \dfrac{\Gamma\vdash\varphi[p:= \psi]}{\Gamma\vdash\exists p.\varphi}\qquad
        \dfrac{\Gamma\vdash\exists p.\varphi\quad\Gamma, \varphi\vdash\psi}{\Gamma\vdash\psi} (p\notin FV(\Gamma, \psi)) \]
\end{enumerate}

\end{enumerate}

\section*{Домашнее задание №6: <<Логика второго порядка и система F>>}
\begin{enumerate}
\item Требуется ли свобода для подстановки в правилах с квантором?

    \[ \dfrac{\Gamma\vdash\phi}{\Gamma\vdash\forall p.\phi} (p\notin FV(\Gamma)) \qquad
        \dfrac{\Gamma\vdash\forall p.\phi}{\Gamma\vdash\phi[p:=\theta]} \]

Если да --- приведите пример доказуемой при отсутствии свободы для подстановки, но некорректной формулы. 
Если нет --- предложите доказательство корректности правил при любых подстановках.

\item Пусть $\Gamma\vdash\varphi$. Всегда ли можно перестроить доказательство $\varphi$, добавив ещё одну гипотезу:
$\Gamma,\psi\vdash\varphi$? Если нет, каковы могли бы быть ограничения на $\psi$?

\item Пусть $\Gamma\vdash\forall x.\varphi$. Верно ли тогда, что $\Gamma\vdash\forall y.\varphi[x := y]$? 
Если это неверно в общем случае, возможно, это верно при каких-то ограничениях? В случае наличия ограничений
приведите надлежащие контрпримеры.

\item Перенесите в систему $F$ из бестипового лямбда-исчисления следующие функции (приведите выражение, укажите его тип и докажите его):
\begin{enumerate}
\item инъекции и $case$ (операции с алгебраическим типом);
\item истина, ложь, исключающее или;
\item черчёвский нумерал (он должен иметь тип $\forall\alpha.(\alpha\rightarrow\alpha)\rightarrow(\alpha\rightarrow\alpha)$) и инкремент;
\item возведение в степень: $\lambda m.\lambda n.n\ m$;
\item вычитание единицы (трюк зуба мудрости) и вычитание.
\end{enumerate}

\item Напомним определения с лекции:

 	$$\infer{\Gamma\vdash (\text{pack } M, \theta \text{ to } \exists \alpha . \varphi) : \exists \alpha.\varphi}{\Gamma \vdash M : \varphi[\alpha := \theta]}
 	 \quad\infer[\alpha \notin FV(\Gamma, \psi)]{\Gamma \vdash \text{abstype } \alpha \text{ with } x:\varphi \text{ in } M \text{ is } N:\psi}{\Gamma \vdash M : \exists \alpha . \varphi\qquad\Gamma, x : \varphi \vdash N : \psi}
	$$

Покажите, что pack и abstype могут быть заданы так:

	$$	\text{\textbf{pack} } M, \theta \text{ \textbf{to} } \exists \alpha . \varphi =
		\Lambda \beta . \lambda x^{\forall \alpha . \varphi \to \beta} . x \theta M $$
	$$	\text{\textbf{abstype} } \alpha \text{ \textbf{with} } x:\varphi \text{ \textbf{in} } M \text{ \textbf{is} } N:\psi =
		M \psi (\Lambda \alpha . \lambda x ^ \varphi . N)
	$$

То есть, соответствующие правила вывода будут выполнены для так заданных выражений.
\item У правил вывода с кванторами для системы $F$ есть ограничения. Покажите, что эти ограничения существенны: что без них 
с помощью кванторов можно типизировать лямбда-выражения, разрушающие систему $F$ (то есть, нарушающие какие-то её существенные
свойства, например, делающие её противоречивой).
\item \begin{enumerate}
\item Разработайте интерфейс и реализацию для абстрактного типа данных <<множество>> (функции создания пустого множества, 
добавления, удаления, проверки наличия элемента в множестве). Напишите тестовую программу, использующую данный АТД.
\item Сделайте по АТД <<множество>> соответствующий экзистенциальный тип в системе F, перенесите его в Хаскель и дайте
реализацию этого типа. Приспособьте тест из предыдущего пункта.
\end{enumerate}

\item Переформулируйте систему F в исчислении по Карри: укажите новые схемы аксиом для кванторов всеобщности и существования.
\item Переформулируйте операции abstype и pack для исчисления по Карри, укажите соответствующие им лямбда-выражения
и покажите, что эти выражения соответствуют аксиомам.
\end{enumerate}

%\item задача про введение существования (ограничения на G|-\phi[p:=psi] и G,\forall p.\phi->g |- \phi[p:=psi]; можем
%ли ослабить доказательство формулой?)

\section*{Домашнее задание №7: <<Типовая система Хиндли-Милнера>>}
\begin{enumerate}
\item Приведите правило вывода (обозначавшееся на лекции как $7'$), типизирующее \verb!let! для рекурсивной функции:
\begin{verbatim}
let rec x = A in B
\end{verbatim}
Покажите, что это правило делает систему противоречивой.
\item Покажите, что если два логических выражения в логике второго порядка эквивалентны ($\vdash\varphi\rightarrow\psi$ и
$\vdash\psi\rightarrow\varphi$), то соответствующие типы либо одновременно обитаемы, либо одновременно необитаемы.
\item Покажите, как по значению типа $\forall \alpha.\beta\rightarrow\varphi(\alpha)$ строить значение типа
$\beta \rightarrow \forall \alpha.\varphi(\alpha)$ и наоборот $\alpha \notin FV(\beta)$.
\item Типовая система Хиндли-Милнера типизирована по Чёрчу. Измените правила и язык так, чтобы она стала типизирована
по Карри.

\item \emph{О выразительной силе HM.} Заметим, что список --- это <<параметризованные>> числа в 
аксиоматике Пеано. Число --- это длина списка, а к каждому штриху мы присоединяем какое-то значение.
Операции добавления и удаления элемента из списка --- это операции прибавления и вычитания
единицы к числу.

Рассмотрим тип <<бинарного списка>>:

\begin{verbatim}
type 'a bin_list = Nil | Zero of (('a*'a) bin_list) | One of 'a * (('a*'a) bin_list);;
\end{verbatim}

Если бы такое можно было выразить в типовой системе Хиндли-Милнера, то операция добавления
элемента к списку записалась бы на языке Окамль вот так (сравните с прибавлением 1 к числу
в двоичной системе счисления):

\begin{verbatim}
let rec add elem lst = match lst with
    Nil -> One (elem,Nil)
  | Zero tl -> One (elem,tl)
  | One (hd,tl) -> Zero (add (elem,hd) tl)
\end{verbatim}

\begin{enumerate}
\item Какой тип имеет \verb!add! (обратите внимание на ключевое слово \verb!rec!: для 
точного указания соответствующего лямбда-выражения и вывода типа необходимо использовать Y-комбинатор)?
Считайте, что семейство типов \verb!bin_list 'a! предопределено, и обозначается как $\tau_\alpha$.
Также считайте, что определены функции roll и unroll с надлежащими типами.
\item Какой ранг имеет тип этой функции? Почему этот тип не выразим в типовой системе Хиндли-Милнера?
\item Предложите функцию для удаления элемента списка (головы).
\item Предложите функцию для эффективного соединения двух списков (источник для 
вдохновения --- сложение двух чисел в столбик).
\item Предложите функцию для эффективного выделения $n$-го элемента из списка.
\end{enumerate}

\item Рассмотрим следующий код на Окамле, содержащий определения чёрчевских нумералов
и некоторых простых операций с ними:

\begin{verbatim}
let zero = fun f x -> x;;
let plus1 a = fun f -> fun x -> a f (f x);;
let power m n = n m;;

let two = plus1 (plus1 zero);;
let two2 = fun f x -> f (f x);;

let e  = power two two;;          (* не компилируется *)
let e2 = power two2 two2;;        (* компилируется и работает *)
\end{verbatim}

Поясните, почему:
\begin{enumerate}
\item определение $e2$ компилируется и работает;
\item определение $e$ не компилируется.
\end{enumerate}

Пояснение должно содержать необходимые фрагменты вывода типа в системе Хиндли-Милнера, 
или должно показывать, что нужного вывода типа не существует.

\item Какой ранг имеют экзистенциальный тип и тип монады \verb!ST! из Хаскеля?
\end{enumerate}

\section*{Домашнее задание №8-9: <<Обобщённая типовая система; язык Аренд>>}

\begin{enumerate}
\item Укажите тип (род) в исчислении конструкций для следующих выражений (при необходимости определите
типы используемых базовых операций и конструкций самостоятельно):
\begin{enumerate}
\item В алгебраическом типе \verb!'a option = None | Some 'a! предложите тип (род) для: \verb!Some!,
\verb!None! и \verb!option!.
\item Пусть задан род $\textbf{nonzero}: \star\rightarrow\star$, выбрасывающий нулевой элемент из
типа. Например, $\textbf{nonzero}\ \textbf{unsigned}$ --- тип положительных целых чисел.
Тогда, для кода
\begin{verbatim}
template<typename T, T x>
struct NonZero { const static std::enable_if_t<x != T(0), T> value = x; };
\end{verbatim}
предложите тип (род) поля value.
\end{enumerate}

\item Предложите выражение на языке C++ (возможно, использующее шаблоны), имеющее следующий род (тип):
\begin{enumerate}
\item $\star\rightarrow\star\rightarrow\star$; $\ \star\rightarrow\textbf{unsigned}$
\item $\textbf{int}\rightarrow(\star\rightarrow\star)$
\item $(\star\rightarrow\textbf{int})\rightarrow\star$
\item $\Pi x^\star.n^\textbf{int}.F(n,x)$, где $$F(n,x) = \left\{\begin{array}{ll}\textbf{int}, & n = 0\\
                                   x\rightarrow F(n,x), & n > 0\end{array}\right.$$
\end{enumerate}

\item Определите функции из следующих частей $\lambda$-куба (в обобщённой типовой системе) и докажите их тип:
\begin{enumerate}
\item $(\square,\star)$
\item $(\star,\square)$
\item $(\square,\square)$
\end{enumerate}

\item Рассмотрим правый дальний нижний угол $\lambda$-куба ($\{(\star,\star);(\star,\square);(\square,\star)\}$).
Можно предположить, что тогда в такой системе возможны и функции рода $f: \star\rightarrow\star$ 
(как композиция функций $p: \star\rightarrow\alpha$ и $q: \alpha\rightarrow\star$ ---
например, можно кодировать тип его именем, затем по имени типа восстанавливать сам тип обратно).
Почему всё-таки такие функции в обобщённых типовых системах невозможны без четвёртого элемента $(\square,\square)$?

\item Как отмечалось на занятии, мы рассматриваем множество натуральных чисел как множество с дискретной топологией
(для того, чтобы функции из $Nat$ в $Nat$ были бы непрерывны). Поясните, почему дискретная топология гарантирует
непрерывность любой такой функции? Напомним, что непрерывная функция --- такая, у которой любой прообраз открытого
множества открыт.

\item Какова должна быть топология на множестве
пар натуральных чисел (интуитивно мы будем понимать эти пары как рациональные числа, пары <<числитель-знаменатель>>), 
чтобы непрерывными были бы те и только те функции, для которых выполнено $f(p,q) = f(p',q')$ для всех 
таких $p,p',q$ и $q'$, что $p\cdot q' = p'\cdot q$. Напомним, что равенство мы понимаем как наличие непрерывного пути 
между точками.

\item Докажите, приведя компилирующуюся программу на языке Аренд (возможно, вам потребуются функции и приёмы, 
изложенные в документации по языку: \url{https://arend-lang.github.io/documentation/tutorial/PartI/}):
\begin{enumerate}
\item ассоциативность сложения;
\item коммутативность сложения;
\item коммутативность умножения;
\item дистрибутивность: $(a + b)\cdot c = a\cdot c + b \cdot c$;
\item куб суммы: $(a + 1)^3 = a^3 + 3\cdot a^2 + 3 \cdot a + 1$.
\end{enumerate}

\item Определим, что $x$ делится на $p$, если \verb!\Sigma (q : Nat) (p * q = x)!.
\begin{enumerate}
\item Покажите, что если $x$ делится на 6, то $x$ делится и на 3;
\item Покажите, что $x!$ делится на $x$;
\item Покажите, что если $x$ делится на $y$ и $y$ делится на $z$, то $x$ делится на $z$;
\end{enumerate}

\item Определите предикат (т.е. функцию с надлежащим типом) для простого числа \verb!isPrime!.
Покажите, что:
\begin{enumerate}
\item 3 и 11 --- простые числа;
\item Произведение простых чисел непросто;
\item 2 --- единственное чётное простое число.
\end{enumerate}
\end{enumerate}

\end{document}
