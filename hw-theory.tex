\documentclass[10pt,a4paper,oneside]{article}
\usepackage[utf8]{inputenc}
\usepackage[english,russian]{babel}
\usepackage{amsmath}
\usepackage{amsthm}
\usepackage{amssymb}
\usepackage{enumerate}
\usepackage{stmaryrd}
\usepackage{cmll}
\usepackage{mathrsfs}
\usepackage{hyperref}
\usepackage[left=2cm,right=2cm,top=2cm,bottom=2cm,bindingoffset=0cm]{geometry}
\usepackage{proof}
\usepackage{tikz}
\usepackage{multicol}

\makeatletter
\newcommand{\dotminus}{\mathbin{\text{\@dotminus}}}

\newcommand{\@dotminus}{%
  \ooalign{\hidewidth\raise1ex\hbox{.}\hidewidth\cr$\m@th-$\cr}%
}
\makeatother

\usetikzlibrary{arrows,backgrounds,patterns,matrix,shapes,fit,calc,shadows,plotmarks}

\newtheorem{definition}{Определение}
\begin{document}

\begin{center}{\Large\textsc{\textbf{Теоретические домашние задания}}}\\
             \it Теория типов, ИТМО, М3235-М3239, осень 2022 года\end{center}

\section*{Домашнее задание №1: <<вводная лекция>>}

\begin{enumerate}

\item Напомним определения с лекций:

\begin{tabular}{lll}
Обозначение & лямбда-терм & название\\\hline
$T$ & $\lambda a.\lambda b.a$ & истина\\
$F$ & $\lambda a.\lambda b.b$ & ложь\\
$Not$ & $\lambda x.x\ F\ T$ & отрицание\\
$And$ & $\lambda x.\lambda y.x\ y\ F$ & конъюнкция
\end{tabular}

Постройте лямбда-выражения для следующих булевских выражений:
\begin{enumerate}
\item Штрих Шеффера (<<и-не>>)
\item Стрелка Пирса (<<или-не>>)
\item Мажоритарный элемент от трёх аргументов (результат <<истина>>, если истинны не менее двух аргументов)
\end{enumerate}

\item Напомним определения с лекций:

$$f^{(n)}\ X ::= \left\{\begin{array}{ll} X, & n=0\\
                                f^{(n-1)}\ (f\ X), & n>0\end{array}\right.$$

\begin{center}\begin{tabular}{lll}
Обозначение & лямбда-терм & название\\\hline
$\overline{n}$ & $\lambda f.\lambda x.f^{(n)}\ x$ & чёрчевский нумерал\\
$(+1)$ & $\lambda n.\lambda f.\lambda x.n\ f\ (f\ x)$ & прибавление 1\\
$IsZero$ & $\lambda n.n\ (\lambda x.F)\ T$ & проверка на 0
\end{tabular}\end{center}

\begin{center}\begin{tabular}{lll}
Обозначение & лямбда-терм & название\\\hline
$MkPair$ & $\lambda a.\lambda b.(\lambda x.x\ a\ b)$ & создание пары\\
$PrL$ & $\lambda p.p\ T$ & левая проекция\\
$PrR$ & $\lambda p.p\ F$ & правая проекция\\\hline
$Case$ & $\lambda l.\lambda r.\lambda c.c\ l\ r$ & case для алгебраического типа\\
$InL$ & $\lambda l.(\lambda x.\lambda y.x\ l)$ & левая инъекция\\
$InR$ & $\lambda r.(\lambda x.\lambda y.y\ r)$ & правая инъекция\\
\end{tabular}\end{center}

Используя данные определения, постройте выражения для следующих операций над числами:

\begin{enumerate}
\item Вычитание 1
\item Вычитание
\item Деление
\item Сравнение двух чисел ($IsLess$) — истина, если первый аргумент меньше второго
(могут потребоваться пары и/или вычитания)
\item Делимость
\end{enumerate}

\item Проредуцируйте выражение и найдите его нормальную форму: 
\begin{enumerate}
\item $\overline{2}\ \overline{2}$
\item $\overline{2}\ \overline{2}\ \overline{2}$
\item $\overline{2}\ \overline{2}\ \overline{2}\ \overline{2}\ \overline{2}\ \overline{2}\ \overline{2}$
\end{enumerate}

\item Напомним определение Y-комбинатора: $\lambda f.(\lambda x.f\ (x\ x))\ (\lambda x.f\ (x\ x))$.
\begin{enumerate}
\item Покажите, что выражение $Y\ f$ не имеет нормальной формы;
\item Покажите, что выражение $Y\ (\lambda f.\overline{0})$ имеет нормальную форму.
\item Покажите, что выражение $Y\ (\lambda f.\lambda x.(IsZero\ x)\ \overline{0}\ (f\ Minus1\ x))\ 2$ имеет нормальную форму.
\item Какова нормальная форма выражения $Y\ (\lambda f.\lambda x.(IsZero\ x)\ \overline{0}\ ((+1)\ (f\ Minus1\ x)))\ \overline{n}$?
\item Какова нормальная форма выражения $Y\ (\lambda f.\lambda x.(IsZero\ x)\ \overline{1}\ (Mul2\ (f\ Minus1\ x)))\ \overline{n}$?
\item Определите с помощью $Y$-комбинатора функцию для вычисления $n$-го числа Фибоначчи.
\end{enumerate}

\item Определим на языке Хаскель следующую функцию: \verb!show_church n = show (n (+1) 0)!
Убедитесь, что \verb!show_church (\f -> \x -> f (f x))! вернёт 2. 
Пользуясь данным определением и его идеей, реализуйте следующие функции:

\begin{enumerate}
\item \verb!int_to_church! --- возвращает чёрчевский нумерал (т.е. функцию от двух аргументов) по целому числу.
Каков точный тип результата этой функции?
\item сложение двух чёрчевских нумералов.
\item умножение двух чёрчевских нумералов.
\item можно ли определить вычитание 1 и вычитание? Что получается, а что --- нет?
\end{enumerate}

\item На лекции было использовано понятие свободы для подстановки. 
\begin{enumerate}
\item Найдите лямбда-выражение, которое при однократной редукции требует переименования связанных переменных
(редукция невозможна без переименования). 
\item Заметим, что даже если мы запретим использовать одни и те же переменные в разных лямбда-абстракциях,
это не будет решением проблемы переименований. Предложите лямбда-выражение, в котором (а) все лямбда-абстракции
указаны по разным переменным; но (б) через некоторое количество редукций потребуется переименование
связанных переменных.
\end{enumerate}

\item Дадим определение: комбинатор --- лямбда-выражение без свободных переменных.

Также напомним определение:
$$\begin{array}{l}
S := \lambda x.\lambda y.\lambda z.x\ z\ (y\ z)\\
K := \lambda x.\lambda y.x\\
I := \lambda x.x
\end{array}$$

Известна теорема о том, что для любого комбинатора $X$ можно найти выражение $P$
(состоящее только из скобок, пробелов и комбинаторов $S$ и $K$), что $X =_\beta P$.
Будем говорить, что комбинатор $P$ \emph{выражает} комбинатор $X$ в базисе $SK$.

Выразите в базисе $SK$:
\begin{enumerate}
\item $F = \lambda x.\lambda y.y$
\item $\overline{1}$
\item $Not$
\item $Xor$
\item $InL$
\end{enumerate}

\item Чёрчевские нумералы соответствуют натуральным числам в аксиоматике Пеано.
\begin{enumerate}
\item Предложите <<двоичные нумералы>> --- способ кодирования чисел, аналогичный двоичной системе 
(такой, при котором длина записи числа соответствует логарифму числового значения).
\item Предложите реализацию функции (+1) в данном представлении.
\item Предложите реализацию лямбда-выражения преобразования числа из двоичного нумерала в чёрчевский.
\item Предложите реализацию функции сложения в данном представлении.
\item Предложите реализацию функции вычитания в данном представлении.
\item Какова вычислительная сложность арифметопераций с двоичными нумералами?
\end{enumerate}

\end{enumerate}

\section*{Домашнее задание №2: ещё о бестиповом лямбда-исчислении}

\begin{enumerate}
\item Бесконечное количество комбинаторов неподвижной точки. Дадим следующие определения
$$\begin{array}{l}
L := \lambda abcdefghijklmnopqstuvwxyzr.r(thisisafixedpointcombinator)\\
R := LLLLLLLLLLLLLLLLLLLLLLLLLL\end{array}$$
В данном определении терм $R$ является комбинатором неподвижной точки: каков бы ни был терм
$F$, выполнено $R\ F =_\beta F\ (R\ F)$.
\begin{enumerate}
\item Докажите, что данный комбинатор --- действительно комбинатор неподвижной точки.
\item Пусть в качестве имён переменных разрешены русские буквы. Постройте аналогичное выражение
по-русски: с 33 параметрами и осмысленной русской фразой в терме $L$; покажите, что оно является
комбинатором неподвижной точки.
\end{enumerate}

\item Напомним определение аппликативного порядка редукции:
редуцируется самый левый из самых вложенных редексов. Например, в случае выражения
$(\lambda x.I\ I)\ (\lambda y.I\ I)$ самые вложенные редексы --- применения $I\ I$:

$$(\lambda x.\underline{I\ I})\ (\lambda y.\underline{I\ I})$$

и надо выбрать самый левый из них:

$$(\lambda x.\underline{I\ I})\ (\lambda y.I\ I)$$
\begin{enumerate}
\item Проведите аппликативную редукцию выражения $2\ 2$.
\item Постройте выражение, использующее $Y$-комбинатор для вычисления факториала. 
Возможно ли его аппликативное вычисление, или оно не сможет завершиться?
\item Найдите лямбда-выражение, которое редуцируется медленнее при нормальном порядке редукции,
чем при аппликативном, даже при наличии мемоизации.
\end{enumerate}

\item Будем говорить, что выражение $A$ находится в \emph{слабой заголовочной нормальной форме} (WHNF),
если оно не имеет вид $A \equiv (\lambda x.P)\ Q$ (то есть, самый верхний терм его не является редексом).
Выражение находится в \emph{заголовочной нормальной форме} (HNF), когда его верхний терм --- не редекс и не лямбда-абстракция
с бета-редексами в теле.

Верно ли, что <<нормальность>> формы выражения может в процессе редукции только усиливаться
(никакая --- слабая заголовочная Н.Ф. --- заголовочная Н.Ф. --- нормальная форма)?

\item Заметим, что список в лямбда-выражении можно закодировать с помощью алгебраических типов.
Напишите лямбда-выражение для:
\begin{enumerate}
\item вычисления длины списка;
\item функции $map$ (построение нового списка из результатов применения функции к элементам старого);
\item суммы списка целых чисел.
\end{enumerate} 

\item Базис $SKI$ не единственный. Рассмотрим базис $BCKW$:
$$\begin{array}{l}B = \lambda x.\lambda y.\lambda z.x\ (y\ z)\\
C = \lambda x.\lambda y.\lambda z.x\ z\ y\\
K = \lambda x.\lambda y.x\\
W = \lambda x.\lambda y.x\ y\ y\end{array}$$

\begin{enumerate}
\item Покажите, что базис $BCKW$ позволяет выразить любое выражение из базиса $SKI$.
\item Покажите, что любое выражение из базиса $BCKW$ может быть выражено в базисе $SKI$.
\end{enumerate}
\end{enumerate}

\end{document}
